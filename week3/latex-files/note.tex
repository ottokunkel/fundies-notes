\documentclass[12pt]{report}

\input{./requirements/preamble.tex}
\input{./requirements/macros.tex}
\input{./requirements/letterfonts.tex}
\setlength{\parindent}{0pt}
\graphicspath{ {./assets/} }

\title{\Huge{Some Class}\\Random Examples}
\author{\huge{Your Name}}
\date{}

\begin{document}

\chapter{Combinational Building Blocks (2.8)}
\section{K-maps}
\purple{
Graphical way to find simplified equations.
}

The goal is to circle the largest groups of truth outputs.
For each circle, create a sum of products equation where each term includes the products that change. For example:

\includegraphics[width=5in]{K-map.png}


\newpage

\section{Multiplexers (Mux)}
\purple{
A multiplexer is a combinational circuit that chooses among N inputs with an N-bit \textit{select} signal
}
Below is a k-map and implementation of a Mux

\begin{figure}[h!]
  \centering
  \includegraphics[width=2in]{2-bit-mux.png}
  \includegraphics[width=0.45\textwidth]{mux-kmap.png} % First image
  \hspace{0.05\textwidth} % Adjust space between images (optional)
  \includegraphics[width=0.45\textwidth]{mux-implementation.png} % Second image
\end{figure}

\pagebreak

\section{Larger Multiplexers}
Larger multiplexers can be used as well. We will need a \textit{select} signal of $\lg (n)$ for an $n$:1 input multiplexer

\begin{figure}[h!]
  \centering
  \includegraphics[width=2in]{4-bit-mux.png}
\end{figure}

\noindent There are three ways to implement a mux: Two level logic, Tri-states, and hierarchical

\begin{figure}[h!]
  \centering
  \includegraphics[width=6in]{n-bit-mux-implementation.png}
\end{figure}

\noindent Larger Multiplexers can also be used as \textit{lookup} tables to perform any function and can be reprogrammed with different inputs. 
In general, a $2^n$ input Multiplexer can be "reprogrammed" to perform any n-input logic function. In addition, one strategy for cutting the size in
half. You do this by providing one of the literals of the logic function to the input of the Mux. This allows for $2^{n-1}$ input Mux. 

\pagebreak

\section{Decoders}
\purple{
  A combinational circuit that transforms N-bit inputs to a $2^n$ output
}

A decorder is, in a sense, the opposite of a Mux. It has N inputs, and $2^n$ outputs. The outputs are \textit{one hot}. That is, only one is 
True / 1. 

\begin{figure}[h!]
  \centering
  \includegraphics[width=0.45\textwidth]{2:4-decoder.png} % First image
  \hspace{0.05\textwidth} % Adjust space between images (optional)
  \includegraphics[width=0.45\textwidth]{decoder-imp.png} % Second image
\end{figure}

\noindent Decoders combined with OR gates can be used to create other logic functions. Since they are \textit{one-hot} they represent a single min-term. 
It makes it easy to implement something like an XNOR.  

\begin{figure}[h!]
  \centering
  \includegraphics[width=3in]{XNOR-decoder.png}
\end{figure}

\chapter{Arithmetic Circuits (5.2)}
Arithmetic circuits are combinational logic functions that perform the typical arithmetic operations (addition, subtraction, multiplication, division). 
They are fundemental in modern computing. 

\section{addition}
\subsection{Half Adder}
\purple{
A circuit that adds two bits and has one \textit{carry-out} bit to perform binary addition. 
}
There are 2 inputs for each singular bit and then two outputs.
The output signal $S$ represents the single bit output and the output signal $C_{out}$ is for the Carry-bit for the next position / adder.
\medskip

They can be made from an XOR gate and an AND gate. The $C_{out}$ bit will be used when creating multi-bit adders from these single bit ones. 
\begin{figure}[h!]
  \centering
  \includegraphics[width=1.5in]{half-adder.png}
\end{figure}
\pagebreak

\subsection{Full Adder}
\purple{
  Circuit that, like a half-adder performs binary addition, but also has a carry-in, $C_{in}$, bit
}
It basically adds 3 1-bit inputs in order to output two bits \medskip
\begin{figure}[h!]
  \centering
  \includegraphics[width=.5\textwidth]{full-adder.png}
\end{figure}
\pagebreak
\subsection{Carry Propagate Adder (CPA)}
These circuits can be put together to create a Carry Propagate Adder. This takes two N-bit inputs
and a 1-bit $C_{in}$ to produce a N-bit $S$, and a 1-bit $C_{out}$.

\subsection{Ripple Carry Adder}
\purple{
  A linear chain of 1-bit adders to perform N-bit addition
}
This is the simplest way to create a CPA. Here, you chain together full adders along with 1 Half-adder at the end. 
\begin{figure}[h!]
  \centering
  \includegraphics[width=0.25\textwidth]{Carry-Propagate-adder.png} % First image
  \hspace{0.05\textwidth} % Adjust space between images (optional)
  \includegraphics[width=0.65\textwidth]{Ripple-carry-adder.png} % Second image
\end{figure}

The disadvantage with Ripple Carry Adders is that it becomes slow when N is large. 
The bits must wait for each bit to add before it. 

\smallbreak 
This is the formula for the speed of a ripple carry adder, where $t_{\mathrm{FA}}$ is the delay of the full adder:

\[
  t_{\mathrm{ripple}} = Nt_{\mathrm{FA}}
\]
\pagebreak

\subsection{Carry-Lookahead Adder}
\purple{
  Another way to perform N-bit addition. Divides Adders into blocks to perform addition quicker. 
}
This is another type of CPA, which tries to fix the speed problem of Ripple-carry adders. 
It attempts to divide the adder into blocks and determines the $C_{out}$ as soon as the $C_{in}$ is known.

\bigbreak
\noindent CLA's use $G$ Generate and $P$ Propagate signals to find the $C_{out}$. These signals \underline{do not} depend on previous inputs

\bigskip
\begin{enumerate}
  \item \noindent A $G_i$ signal is made if both $A_i$ and $B_i$ are 1 
  \item \noindent A $P_i$ if a carry if either $A_i$ or $B_i$ are 1
\end{enumerate}
\[
C_i = A_i B_i + (A_i + B_i)C_{i-1} = G_i + P_i C_{i-1}
\]
\noindent In other words, there will be a carry if the $i^{\mathrm{th}}$ column generates or is able to propagates a carry. 
We can rewrite this into blocks.

\[
  G_{3:0} = G_3 + P_3 (G_2 + P_2 (G_1 + P_1 G_0))
\]

\[
  P_{3:0} = P_3 P_2 P_1 P_0
\]

\[
  \boxed{C_i = G_{i:j} + P_{i:j}C_j}
\]

\underline{The key is that the Generate and Propagate signals can be found immediately}
without waiting for the previous column

\begin{figure}[h!]
  \centering
  \includegraphics[width=.5\textwidth]{Carry-lookahead.png}
\end{figure}

Where $t_{\mathrm{pg}}$ is the delay of the (P, G) gates. $t_{\mathrm{pg\_block}}$ is the delay to find the gen/prop signals. $k$ is the bits per block and $N$ is the number of blocks.
\[
  T_{\mathrm{CLA}} = t_{\mathrm{pg}} + t_{\mathrm{pg\_block}} + \left( \frac{N}{k} - 1 \right)t_{\mathrm{AND\_OR}} + kt_{\mathrm{FA}}
\]

\subsection{Prefix Adders}
\purple{
  A way to perform N-bit addition like CLA, but recursively breaks down the N-bits into blocks. 
}
These are the same as the Carry lookahead adders, however they calculate the $G_i$ and $P_i$ signals for each 2 columns, then every 4 columns etc.

\[
  {S}_{i}=(A_{i}\oplus B_{i})\oplus{C}_{i-1}
\]

\begin{figure}[h!]
  \centering
  \includegraphics[width=.7\textwidth]{prefix_adder_schematic.png}
  \includegraphics[width=.7\textwidth]{prefix_schematic_legend.png}
\end{figure}

The critical path of an N-bit prefix adder involves the combination of $P_i$ and $G_i$ followed by $\lg N$ stages of black prefix cells

\[
  t_{PA} = t_{pg} + \lg N \left( t_{pg \_ \mathrm{prefix}} \right) + t_{\mathrm{XOR}}
\]
\pagebreak

\subsection{Subtraction}
\purple{
Using two's complement, it is exactly the same as addition. That is flipping the bits and adding 1.
}

\subsection{Comparators}
\purple{
  

}



\end{document}