\documentclass[12pt]{report}

\input{./requirements/preamble.tex}
\input{./requirements/macros.tex}
\input{./requirements/letterfonts.tex}
\setlength{\parindent}{0pt}
\graphicspath{ {./assets/} }

\title{\Huge{Some Class}\\Random Examples}
\author{\huge{Your Name}}
\date{}

\begin{document}
\chapter*{Chapter 6}

\dfn{Architecture}{
  "A programmer's view of a computer"
  Defined by the instruction set and operand locations (registers and memory) 
}

Here are some examples of popular Computer architecures, most computers use ARM and x86
\begin{multicols}{2}
  \begin{itemize}
    \item x86
    \item MIPS
    \item PowerPC
  \end{itemize}

  \columnbreak

  \begin{itemize}
    \item ARM
    \item SPARC
    
  \end{itemize}
\end{multicols}


\section{Assembly}
\purple{
  assembly is the human-readable representation of a Computer Architecture's native language.
}

\subsection{Instructions}
\purple{
  Assembly is made up of single line instructions that tell the computer what and how to perform different operations
}
\begin{figure}[H]
  \centering
  \includegraphics[width=.9\textwidth]{instructions.png}
\end{figure}
\pagebreak

\subsection{Design Principles}
\purple{
  In this section we will include a few principles to remember. Here are two
}

\underline{\textbf{Principle 1: Simplicity favors regularity}}
- instructions with consistent number of operands is better. In the above image, we use multiple instructions instead of one instruction with multiple operands
\bigbreak
\underline{\textbf{Principle 2: Make the common case fast.}}
- One thing that MIPS does is keeping the number of different operations low. More complex operations are made into multiple instructions. 

\subsection{Operands, Registers, Memory and Constants}

\purple{
Operands are the things upon which the instuctions act on. They can be stored in registers, memory, or constants in the instruction itself.
The instruction sets themselves only hold 32-64 bits
}
\subsection{Registers}
\purple{Memory takes a long time to access so there are Registers to quickly access common operands.}

\smallbreak
MIPS uses 32 registers
\bigbreak
\underline{\textbf{Principle 3: Smaller is faster}}
- For example, its is better to search through a small set of 32 registers than 1000.

\begin{figure}[H]
  \centering
  \includegraphics[width=.9\textwidth]{instructions-with-registers.png}
\end{figure}



\end{document}