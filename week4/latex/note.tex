\documentclass[12pt]{report}

\input{./requirements/preamble.tex}
\input{./requirements/macros.tex}
\input{./requirements/letterfonts.tex}
\usepackage{float}  % Add this line to use the [H] option
\usepackage{subcaption}  % Add this package for subfigures
\setlength{\parindent}{0pt}
\graphicspath{ {./assets/} }

\title{\Huge{Some Class}\\Random Examples}
\author{\huge{Your Name}}
\date{}

\begin{document}
\chapter{Sequential Logic}

\section{Latches and Flip-flops}
\purple{Latches and flip-flops are fundamental building blocks in digital electronics used for storing binary information.}
These bistable devices have two stable states, representing either a binary 0 or 1.

\section{SR Latch}
\purple{The SR (Set-Reset) Latch is a simple storage device that holds one bit of data.} It features:
\begin{itemize}
    \item Two inputs: S (Set) and R (Reset)
    \item Two outputs: Q and Q'
    \item Function: Set input sets output to 1, Reset input resets output to 0
\end{itemize}
\nt{Avoid simultaneous activation of S and R to prevent undefined states}

\begin{figure}[H]
    \centering
    \begin{subfigure}{0.45\textwidth}
        \centering
        \includegraphics[width=\textwidth]{SR_latch_symbol.png}
        \caption{SR Latch Symbol}
    \end{subfigure}
    \hfill
    \begin{subfigure}{0.45\textwidth}
        \centering
        \includegraphics[width=\textwidth]{SR_latch_schematic.png}
        \caption{SR Latch Schematic}
    \end{subfigure}
\end{figure}

\section{D Latch}
\purple{The D (Data or Delay) Latch captures input at specific times.} Key characteristics include:
\begin{itemize}
    \item Single data input (D) and a clock input
    \item When clock is active: Output Q follows input D
    \item When clock is inactive: Output Q holds its last state
\end{itemize}
\nt{D Latches are useful for data synchronization}

\section{D Flip-flop}
\purple{The D Flip-flop is an edge-triggered device ideal for synchronous circuits.} Characteristics include:
\begin{itemize}
    \item Changes state only at clock signal edge (rising or falling)
    \item Captures D input value at clock edge
    \item Holds value until next clock edge
\end{itemize}
\nt{D Flip-flops ensure precise timing in digital systems}

\begin{figure}[H]
    \centering
    \begin{subfigure}{0.45\textwidth}
        \centering
        \includegraphics[width=\textwidth]{D-flip-flop_symbol.png}
        \caption{D Flip-flop Symbol}
    \end{subfigure}
    \hfill
    \begin{subfigure}{0.45\textwidth}
        \centering
        \includegraphics[width=\textwidth]{D_flip-flop_schematic.png}
        \caption{D Flip-flop Schematic}
    \end{subfigure}
\end{figure}

\section{Register}
\purple{A Register is composed of multiple flip-flops and is used to store and manipulate data.} Features include:
\begin{itemize}
    \item Stores multiple bits of data
    \item Applications: data storage, transfer, and manipulation
    \item Configurable for operations like shifting, loading, and clearing
\end{itemize}

\begin{figure}[H]
    \centering
    \includegraphics[width=0.8\textwidth]{Register_schematic_and_symbol.png}
    \caption{Register Schematic and Symbol}
\end{figure}

\section{Enabled Flip-flop}
\purple{The Enabled Flip-flop adds control over data acceptance.} It includes:
\begin{itemize}
    \item An enable input
    \item Enable active: Captures input data at clock edge
    \item Enable inactive: Retains current state regardless of input
\end{itemize}

\section{Resettable Flip-flop}
\purple{A Resettable Flip-flop allows forced state changes.} It features:
\begin{itemize}
    \item A reset input
    \item Can force flip-flop to a known state (typically 0)
\end{itemize}
\nt{Resettable Flip-flops are useful for circuit initialization during power-up or specific conditions}

\begin{figure}[H]
    \centering
    \includegraphics[width=0.8\textwidth]{Resettable_flip_flop_schematic_and_symbol.png}
    \caption{Resettable Flip-flop Schematic and Symbol}
\end{figure}

\section{Counter}
\purple{Counters are sequential circuits for tracking events.} Characteristics include:
\begin{itemize}
    \item Progresses through predetermined state sequence
    \item Implemented using flip-flops
    \item Can count up, down, or in specific patterns
\end{itemize}
\nt{Counters are used for event occurrence counting}

\section{Shift Register}
\purple{Shift Registers enable serial data manipulation.} Features include:
\begin{itemize}
    \item Allows serial shifting of data in or out
    \item Composed of flip-flops in a chain
    \item Applications: data storage, transfer, and format conversion
\end{itemize}

\begin{figure}[H]
    \centering
    \begin{subfigure}{0.45\textwidth}
        \centering
        \includegraphics[width=\textwidth]{shift-register_symbol.png}
        \caption{Shift Register Symbol}
    \end{subfigure}
    \hfill
    \begin{subfigure}{0.45\textwidth}
        \centering
        \includegraphics[width=\textwidth]{shift_register_schematic.png}
        \caption{Shift Register Schematic}
    \end{subfigure}
\end{figure}

\subsection{Scan Chains}
\purple{Scan Chains are crucial for digital circuit testing.} They are:
\begin{itemize}
    \item Series of flip-flops connected in a chain
    \item Enables shifting of test data into and out of the circuit
\end{itemize}
\nt{Scan Chains are used to test internal circuit states and ensure correct operation}

\end{document}
